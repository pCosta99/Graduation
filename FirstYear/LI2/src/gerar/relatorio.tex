\documentclass{article}
\usepackage[utf8]{inputenc}

\title{Relatório sobre gerador de estados aleatorios}
\date{03-06-2018}
\author{Pedro Costa, Pedro Barbosa, Ricardo Veloso}

\begin{document}
  \pagenumbering{gobble}
  \maketitle
  \newpage
  \pagenumbering{arabic}

\section{Gerador}

Nesta única secção abordam-se vários tópicos relativos ao método de gerar estados aleatórios com uma única solução.

\subsection{Falhas}

Devido a problemas de eficiência (conjuntamente com problemas de tempo) não nos foi possível gerar estados com casas bloqueadas.
No entanto, fazer esta abordagem não seria problemático, abordaremos este tópico mais tarde.

\subsection{Ponto de partida}

Antes de mais, convém referir que o nosso solver não nos indica apenas uma solução. O solver que desenvolvemos é capaz de determinar
todas as soluções que um estado contém. Desta maneira, a primeira coisa que fazemos é criar um estado vazio, com as dimensões introduzidas
pelo utilizador e, a partir daí, obtemos todas as soluções existentes para esse estado. Assim, temos um array com X estados todos diferentes, 
de onde iremos escolher um (também ao acaso). Isto permite tornar a geração mais aleatória. 

\subsection{Estados fáceis}

Há uma ligeira diferença na abordagem para gerar um estado fácil e um estado difícil. 
Quanto a um estado fácil, a nossa decisão foi partir de um estado completo (como referido anteriormente, escolhemos um de entre as várias 
resoluções possíveis do vazio com o tamanho correto) e desconstruí-lo até um certo ponto (60 por cento do estado original). Para este efeito, 
tudo o que fazemos é simplesmente retirar uma casa ao acaso e verificar se o estado continua a ser resolúvel através, unicamente, de ajudas lógicas.
Fazendo isto até atingir o ponto de desconstrução desejado, conseguimos gerar estados fáceis conforme era suposto.

\subsection{Estados dificeis}

Quantos aos estados difíceis, tal como referimos, a abordagem é ligeiramente diferente. Mais uma vez partimos de um estado vazio e das suas possíveis
soluções, mas, desta vez, o algoritmo é um pouco mais recorrente à força bruta. Isto significa que sorteamos várias posições do estado completo (também este
escolhido aleatoriamente, como mencionado na subsecção 1.2) e atribuímo-las ao estado vazio. De seguida, utilizamos o nosso solver para verificar se o estado gerado
tem uma única solução. Caso isto se verifique retornamos este estado, caso contrário repetimos o processo até obter um estado de solução única.
Para estados difíceis, o estado gerado apenas contém 10 por cento da solução final.


\subsection{Pormenores que aumentam o fator aleatório do nosso gerador}
\begin{itemize}
	\item Utilizar uma das várias possíveis soluções de um estado vazio, escolhida também aleatoriamente.
	\item Sempre que sorteamos uma casa para preencher, na verdade, sorteamos 5 e escolhemos aleatoriamente uma dessas. 
	\item A cada vez que se volta a sortear o estado, caso não se obtenha um válido numa tentativa, volta-se a escolher aleatoriamente dentro das possíveis soluções.
\end{itemize}

\subsection{Como implementar bloqueadas}
Após criar o estado com solução única seria apenas necessário colocar um certo número de bloqueadas aleatoriamente até que se obtesse uma solução única.

\end{document}